\chapter{Referencial Teórico} \label{refteor}

\textcolor{red}{Este capítulo falará do referencial teórico. ele será dividido em x partes.}

\section{Detecção de bordas}

Esta seção abordará os conceitos relacionados à detecção de bordas, assim como as técnicas de detecção utilizadas na literatura.

\subsection{Conceitos Iniciais}
%definicao de borda

Borda é considerada a mudança abrupta ou descontinuidade em uma ou mais propriedades de sinais bidimensionais, como imagens, entre áreas vizinhas \cite{li2009markov}. Exemplos de propriedades das imagens são a intensidade, no caso de imagens em preto e branco, e os valores de R, G e B no caso de imagens coloridas.

É importante detectar estas descontinuidades pois elas podem corresponder à borda do objeto retratado, bem como mudanças na sua profundidade, iluminação ou textura \cite{descontinuidades}.

%definicao de deteccao de borda

A detecção de bordas busca, portanto, identificar onde ocorre esta descontinuidade. Sua saída é chamada mapa de bordas, uma matriz com o mesmo tamanho da imagem de origem, cujos pixeis têm valor 1 caso aquela posição, na imagem, seja correspondente a uma borda e 0 caso não seja.

Em processamento de imagens, a detecção de bordas possui as mais diversas aplicações, como por exemplo detecção de tumores no cérebro \cite{detectumor}, estudos sobre o comportamento dos pombos \cite{pigeon} e navegação espacial \cite{navegespacial}.
% LE ISSO http://citeseerx.ist.psu.edu/viewdoc/download;jsessionid=8F53C8A13258AFE7901EECFF8DFAC361?doi=10.1.1.27.1821&rep=rep1&type=pdf

\subsection{Tipos de Detectores}

%metodos de deteccao


\section{Detecção de linhas}

\subsection{Conceitos Iniciais}

\subsection{Técnicas de Detecção \textit{mudar este nome}}

\section{Reconhecimento de Retângulos}

\subsection{Conceitos Iniciais}

\subsection{Técnicas de Reconhecimento}

Ok eu não sei se vai ter isso mas acho uma boa

\section{Inspeção Automática de Monitores}

\subsection{Conceitos Iniciais}

\subsection{Técnicas de Inspeção}

