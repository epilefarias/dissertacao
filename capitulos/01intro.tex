\chapter{Introdução}

%Segundo a norma de formata{\c c}\~ao de teses e disserta{\c c}\~oes do
%Instituto Alberto Luiz Coimbra de P\'os-gradua{\c c}\~ao e Pesquisa de
%Engenharia (COPPE), toda abreviatura deve ser definida antes de
%utilizada.\abbrev{COPPE}{Instituto Alberto Luiz Coimbra de P\'os-gradua{\c
%c}\~ao e Pesquisa de Engenharia}

%Do mesmo modo, \'e imprescind\'ivel definir os s\'imbolos, tal como o
%conjunto dos n\'umeros reais $\mathbb{R}$ e o conjunto vazio $\emptyset$.
%\symbl{$\mathbb{R}$}{Conjunto dos n\'umeros reais}
%\symbl{$\emptyset$}{Conjunto vazio}

1 - Introdução: fundamentar o problema, mostrar a sua importância, que é a detecção de telas para avaliação automática de qualidade/imagem/conformidade ou identificação de dispositivo, e introduzir as contribuições;

contextualizar o tema

o que é o problema?
detecção de erro em monitores.
PERGUNTA DE PESQUISA *importante*

objetivos
 - geral
 - específicos
 
justificativa
 - qual a importancia do problema?
 - como é feito atualmente?
 - quais as vantagens de automatizar?
 - porque fazer nessa etapa?
 - o quanto o problema custa?
 - qual minha contribuição nesse aspecto? (metodologia)

fase metodológica da pesquisa (pergunta nanda)

%------refteorico
%reconhecimento de padroes é uma area de grande abrangência

%reconhecimento de padrões e visao computacional estão relacionados

%recentemente, vem crescendo o interesse da comunidade científica em RP atraves de filtragem

%o método da filtragem discriminativa foi desenvolvido...

%-----metodologia
%uma possibilidade, sendo uma das contribuições desta tese, é usar DF pra detectar um padrão e suas variações

%a segunda contribuição desta dissertação é....

%-----experimentos
%os desempenhos dos métodos propostos foram avaliados no contexto de um problema de detecção de face humana


 
 \section{Organização da Dissertação}
 
 se puder fazer um diagrama seria massa