\chapter{Detec{\c c}\~ao de telas}

%Para ilustrar a completa ades\~ao ao estilo de cita{\c c}\~oes e listagem de
%refer\^encias bibliogr\'aficas, a Tabela~\ref{tab:citation} apresenta cita{\c
%c}\~oes de alguns dos trabalhos contidos na norma fornecida pela CPGP da
%COPPE, utilizando o estilo num\'erico.

%\begin{table}[h]
%\caption{Exemplos de cita{\c c}\~oes utilizando o comando padr\~ao
%  \texttt{\textbackslash cite} do \LaTeX\ e
%  o comando \texttt{\textbackslash citet},
%  fornecido pelo pacote \texttt{natbib}.}
%\label{tab:citation}
%\centering
%{\footnotesize
%\begin{tabular}{|c|c|c|}
%  \hline
%  Tipo da Publica{\c c}\~ao & \verb|\cite| & \verb|\citet|\\
%  \hline
%  Livro & \cite{book-example} & \citet{book-example}\\
%  Artigo & \cite{article-example} & \citet{article-example}\\
%  Relat\'orio & \cite{techreport-example} & \citet{techreport-example}\\
%  Relat\'orio & \cite{techreport-exampleIn} & \citet{techreport-exampleIn}\\
%  Anais de Congresso & \cite{inproceedings-example} &
%    \citet{inproceedings-example}\\
%  S\'eries & \cite{incollection-example} & \citet{incollection-example}\\
%  Em Livro & \cite{inbook-example} & \citet{inbook-example}\\
%  Disserta{\c c}\~ao de mestrado & \cite{mastersthesis-example} &
%    \citet{mastersthesis-example}\\
%  Tese de doutorado & \cite{phdthesis-example} & \citet{phdthesis-example}\\
%  \hline
%\end{tabular}}
%\end{table}

2 - A detecção de telas: Mostrar o que é o problema e como esse processo pode ajudar em ambientes industriais e domésticos;

detecção de telas
detecção de retângulos
como ele se encaixa na detecção de padrões em imagens
particularidades da tela de TV q podem ser utilizadas

como a detecção de tela pode ajudar
aplicação domestica
aplicação industrial
